\section{WIP: HM(X): HM Type System with Constraint System}

\cite{Odersky:1999}\cite{Eekelen:2004}

\subsection{Constraint System}

\begin{definition}[単純制約システム (simple constraint system)]
  単純制約システムとは,以下の組 $(\Omega, \Vdash)$ のこと.
  \begin{itemize}
    \item 非空のアルファベット $\Omega$.
    \item 関係 $({\Vdash}) \subseteq \Power(\Omega) \times \Omega$ で,以下を満たすもの.
    \begin{itemize}
      \item 任意の $C \in \Power(\Omega)$,$P \in C$ について,$C \Vdash P$.
      \item 任意の $C, D \in \Power(\Omega)$,$Q \in \Omega$ について,$(\forall P \in D\ldotp C \Vdash P)$ かつ $D \Vdash Q$ ならば $C \Vdash Q$.
    \end{itemize}
  \end{itemize}
  この時,$C \in \Power(\Omega)$ を\emph{制約\emphp{constraint}}と呼ぶ.また,$({\Vdash}) \subseteq (\Power(\Omega))^2$ への拡張を,$C \Vdash D \defrel{\iff} \forall P \in D\ldotp C \Vdash P$ と定義する.$C \Vdash D$ かつ $D \Vdash C$ の時,$C \dashVdash D$ と表記する.さらに,$C \land D = C \cup D$ と表記する.
\end{definition}

\begin{proposition}
  単純制約システム $(\Omega, \Vdash)$ は,以下を admissible にする.
  \begin{gather*}
    \infer{C \Vdash C}{}
    \\
    \infer{C_1 \Vdash C_3}{
      C_1 \Vdash C_2
      &
      C_2 \Vdash C_3
    }
    \\
    \infer{C \land C' \Vdash D}{
      C \Vdash D
    }
  \end{gather*}
\end{proposition}
\begin{proof}
  \begin{align*}
    C \Vdash C &\iff \forall P \in C\ldotp C \Vdash P \\
    C_1 \Vdash C_2 \land C_2 \Vdash C_3
    &\implies \forall Q \in C_3\ldotp C_1 \Vdash C_2 \land C_2 \Vdash Q \\
    &\implies \forall Q \in C_3\ldotp (\forall P \in C_2\ldotp C_1 \Vdash P) \land C_2 \Vdash Q \\
    &\implies \forall Q \in C_3\ldotp C_1 \Vdash Q &(\because \text{単純制約システムの公理}) \\
    &\implies C_1 \Vdash C_3 \\
    \forall P \in C \land C'\ldotp C \in P &\implies C \land C' \Vdash C \\
    C \Vdash D &\implies C \land C' \Vdash C \land C \Vdash D \implies C \land C' \Vdash D
  \end{align*}
  より明らか.
\end{proof}

\begin{definition}[Cylindric 制約システム (cylindric constraint system)]
  Cylindric 制約システムとは,以下の組 $(\Omega, \Vdash, \mathcal{A}, \exists)$ のこと.
  \begin{itemize}
    \item 単純制約システム $(\Omega, \Vdash)$.
    \item 変数の無限集合 $\mathcal{A}$.
    \item 関数の族 $\{\exists \alpha\}_{\alpha \in \mathcal{A}} \in \prod_{\alpha \in \mathcal{A}} \Power(\Omega) \to \Power(\Omega)$ で以下を満たすもの.
    \begin{itemize}
      \item 任意の $C \in \Power(\Omega)$,$\alpha \in \mathcal{A}$ について,$C \Vdash \exists \alpha\ldotp C$.
      \item 任意の $C, D \in \Power(\Omega)$,$\alpha \in \mathcal{A}$ について,$C \Vdash D$ ならば,$\exists \alpha\ldotp C \Vdash \exists \alpha\ldotp D$.
      \item 任意の $C, D \in \Power(\Omega)$,$\alpha \in \mathcal{A}$ について,$\exists \alpha\ldotp (C \land \exists \alpha\ldotp C) \dashVdash (\exists \alpha\ldotp C) \land (\exists \alpha\ldotp D)$.
      \item 任意の $C \in \Power(\Omega)$,$\alpha, \beta \in \mathcal{A}$ について,$\exists \alpha\ldotp \exists \beta\ldotp C \dashVdash \exists \beta\ldotp \exists \alpha\ldotp C$.
    \end{itemize}
    ただし,$\exists \alpha\ldotp C = (\exists \alpha)(C)$ である.
  \end{itemize}
\end{definition}

\begin{definition}[自由変数]
  Cylindric 制約システム $(\Omega, \Vdash, \mathcal{A}, \exists)$,制約 $C \in \Power(\Omega)$ について,自由変数の集合を $\mathrm{fv}(C) = \{\alpha \mid \exists \alpha\ldotp C \not\dashVdash C\}$ とおく.
\end{definition}

\begin{definition}[充足可能 (satisfiable)]
  Cylindric 制約システム $(\Omega, \Vdash, \mathcal{A}, \exists)$,制約 $C \in \Power(\Omega)$ について,$\Vdash \exists \mathrm{fv}(C)\ldotp C$ の時,$C$ は充足可能であるという.
\end{definition}

\begin{lemma}
  Cylindric 制約システム $(\Omega, \Vdash, \mathcal{A}, \exists)$,制約 $C \in \Power(\Omega)$ について,以下は同値.
  \begin{itemize}
    \item $C$ は充足可能.
    \item $\exists \alpha\ldotp C$ は充足可能.
  \end{itemize}
\end{lemma}

\begin{definition}[項制約システム (term constraint system)]
  項制約システムとは,
  \begin{itemize}
    \item 項代数 $(\Sigma, X)$.
    \item 述語のランク付きアルファベット $P$.
    \item Cylindric 制約システム $(\Omega, \Vdash, X, \exists)$,ただし,$\Omega = \{p(\tau_1, \ldots, \tau_n)\mid p^{(n)} \in P, \tau_1, \ldots, \tau_n \in \semanticf{(\Sigma, X)}\}$.
  \end{itemize}
  の組 $(\Sigma, P, \Omega, \Vdash, X, \exists)$ で,以下を満たすもの.
  \begin{itemize}
    \item 任意の $\alpha \in X$ について,$\Vdash \alpha = \alpha$.
    \item 任意の $\alpha_1, \alpha_2 \in X$ について,$(\alpha_1 = \alpha_2) \Vdash (\alpha_2 = \alpha_1)$.
    \item 任意の $\alpha_1, \alpha_2, \alpha_3 \in X$ について,$(\alpha_1 = \alpha_2) \land (\alpha_2 = \alpha_3) \Vdash (\alpha_1 = \alpha_3)$.
    \item 任意の $\alpha_1, \alpha_2 \in X$,$C \in \Power(\Omega)$ について,$(\alpha_1 = \alpha_2) \land \exists \alpha_1\ldotp (C \land (\alpha_1 = \alpha_2)) \Vdash C$.
    \item 任意のコンテキスト $T[] \in \mathcal{C}(\mathcal{T})$,$\tau_1, \tau_2 \in \semanticf{(\Sigma, X)}$ について,$(\tau_1 = \tau_2) \Vdash (T[\tau_1] = T[\tau_2])$.
    \item 任意の $P \in \Omega$,$\tau \in \semanticf{(\Sigma, X)}$,$\alpha \in X$,$\alpha \not\in \mathit{fv}(\tau)$ について,$P[\alpha \assign \tau] \dashVdash \exists \alpha\ldotp (P \land (\alpha = \tau))$.
  \end{itemize}
\end{definition}

\begin{lemma}[Renaming]
  項制約システム $(\Sigma, P, \Omega, \Vdash, X, \exists)$,$C \in \Power(\Omega)$,$\alpha_1, \alpha_2 \in X$ について,$\alpha_2$ が $C$ に出現しない時,$\exists \alpha_1\ldotp C \dashVdash \exists \alpha_2\ldotp C[\alpha_1 \assign \alpha_2]$.
\end{lemma}
