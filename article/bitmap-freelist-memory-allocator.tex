\section{Memory Allocator with BitMap Free List}

\cite{Ueno:2011}\cite{Ueno:2016}

\subsection{Heap Structure}

\begin{definition}[セグメント (segment)]
  セグメントとは、以下による組 $S = (M, L)$ のことである:
  \begin{description}
    \item[$M$] ビットマップ。
    \item[$L$] ブロック配列。
  \end{description}
  セグメントのクラスを $\mathrm{Seg}$ と表記する。
\end{definition}

サブヒープは、$N_c$ 個のクラスによるヒープ分割領域であり、それぞれのクラス $i$ はブロックサイズ $\mathrm{sizeOfClass}(i)$ を持ち、$\forall i_1 < i_2\ldotp \mathrm{sizeOfClass}(i_1) < \mathrm{sizeOfClass}(i_2)$ を満たす。
\begin{definition}[サブヒープ (sub-heap)]
  クラス $i$ のサブヒープとは、以下による組 $V_i = (R)$ のことである:
  \begin{description}
    \item[$R \in \mathbb{N}^*$] 空きセグメント番号の列。$\mathrm{seg}(V_i) = R$ と表記する。
  \end{description}
\end{definition}

\begin{definition}[ヒープ (heap)]
  ヒープとは、以下による組 $H = (A, \{V_i\}_{i \in [N_c]}, F)$ のことである:
  \begin{description}
    \item[$A \in \mathrm{Seg}^*$] セグメントの列。
    \item[$\{V_i\}_{i \in [N_c]}$] サブヒープの族。$\mathrm{subheap}_i(H) = V_i$ と表記する。
    \item[$F \in \mathrm{Seg}^*$] 空きセグメントの列。$\mathrm{free}(H) = F$ と表記する。
  \end{description}
\end{definition}

\subsection{Initialize}

\begin{center}
  \begin{tabular}{|p{0.9\textwidth}|}
    \hline
    \begin{algorithmic}
      \Ensure{$H$}
      \For{$i \in [N_c]$}
        \State{$V_i \assign (\mathrm{sizeOfClass(i)}, \epsilon)$}
      \EndFor
      \State{$H \assign (\{V_i\}_{i \in [N_c]}, \epsilon)$}
    \end{algorithmic}
    \\\hline
  \end{tabular}
\end{center}

\subsection{Allocation}

\begin{center}
  \begin{tabular}{|p{0.9\textwidth}|}
    \hline
    \begin{algorithmic}
      \Require{$H, s$}
      \Ensure{$H, b$}
      \State{$i = \mathrm{classOfSize}(s)$}
      \If{$i = -1$}
        \State{$b \assign \mathrm{allocFreeSize}(s)$}
      \Else
        \State{$V_i \assign \mathrm{subheap}_i(H)$}
        \If{$|\mathrm{seg}(V_i)| > 0$}
          \State{$S \assign \mathrm{seg}(V_i)(0)$}
        \ElsIf{$|\mathrm{free}(H)| > 0$}
          \State{$S \assign \mathrm{free}(H)(0)$}
        \Else
          \State{$a$}
        \EndIf
        \State{$\mathrm{return}\; 2$}
      \EndIf
    \end{algorithmic}
    \\\hline
  \end{tabular}
\end{center}

\begin{definition}
  \begin{align*}
    \mathrm{classOfSize}(s) &= \left\{\begin{array}{ll}
      -1 &(\forall i \in [N_c]\ldotp \mathrm{sizeOfClass}(i) < s) \\
      \max\{i \in [N_c] \mid s \leq \mathrm{sizeOfClass}(i)\} &(\text{otherwise})
    \end{array}\right.
  \end{align*}
\end{definition}

\subsection{Free}

\begin{center}
  \begin{tabular}{|p{0.9\textwidth}|}
    \hline
    \begin{algorithmic}
      \Require{$H, b$}
      \Ensure{$H$}
      \State{$\mathrm{return}\; 1$}
    \end{algorithmic}
    \\\hline
  \end{tabular}
\end{center}
