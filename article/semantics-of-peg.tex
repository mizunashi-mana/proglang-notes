\section{Syntax and Semantics of PEG}

\cite{Ford:2002}, \cite{Ford:2004}

\subsection{Syntax}

\begin{align*}
  e
  &\Coloneqq \epsilon &(\text{epsilon}) \\
  &\BNFor \sigma &(\text{terminal}) \\
  &\BNFor A &(\text{non-terminal}) \\
  &\BNFor ee &(\text{sequence}) \\
  &\BNFor e \mathrel{/} e &(\text{alternative}) \\
  &\BNFor e^* &(\text{repetition}) \\
  &\BNFor \mathop{!} e &(\text{not predicate}) \\
  \sigma &\in \Sigma \\
  A &\in N
\end{align*}

\begin{definition}
  PEG 文法とは,以下による組 $G = (\Sigma, N, R, e_0)$ のことである.
  \begin{description}
    \item[$\Sigma$] 終端記号の集合.
    \item[$N$] 非終端記号の集合.
    \item[$R$] $A \to e$ を満たす規則の集合.規則は,非終端記号に対して必ず一つ.
    \item[$e_0$] 初期式.
  \end{description}
\end{definition}

\subsection{Structured Semantics}

\begin{gather*}
  \infer{\langle \epsilon, x\rangle \to \epsilon}{}
  \\
  \infer{\langle \sigma, \sigma x\rangle \to \sigma}{}
  \hspace{2em}
  \infer{\langle \sigma, \sigma' x\rangle \symfail}{
    \sigma \neq \sigma'
  }
  \\
  \infer{\langle A, xy\rangle \to x}{
    A \assign e \in R
    &
    \langle e, xy\rangle \to x
  }
  \\
  \infer{\langle e_1e_2, x_1x_2y\rangle \to x_1x_2}{
    \langle e_1, x_1x_2y\rangle \to x_1
    &
    \langle e_2, x_2y\rangle \to x_2
  }
  \hspace{2em}
  \infer{\langle e_1e_2, x\rangle \symfail}{
    \langle e_1, x\rangle \symfail
  }
  \hspace{2em}
  \infer{\langle e_1e_2, x_1y\rangle \symfail}{
    \langle e_1, x_1y\rangle \to x_1
    &
    \langle e_2, y\rangle \symfail
  }
  \\
  \infer{\langle e_1 \mathrel{/} e_2, xy\rangle \to x}{
    \langle e_1, xy\rangle \to x
  }
  \hspace{2em}
  \infer{\langle e_1 \mathrel{/} e_2, xy\rangle \to x}{
    \langle e_1, xy\rangle \symfail
    &
    \langle e_2, xy\rangle \to x
  }
  \\
  \infer{\langle e^*, x_1x_2y\rangle \to x_1x_2}{
    \langle e, x_1x_2y\rangle \to x_1
    &
    \langle e^*, x_2y\rangle \to x_2
  }
  \hspace{2em}
  \infer{\langle e^*, x\rangle \to \epsilon}{
    \langle e, x\rangle \symfail
  }
  \\
  \infer{\langle !e, xy\rangle \to \epsilon}{
    \langle e, xy\rangle \symfail
  }
  \hspace{2em}
  \infer{\langle !e, xy\rangle \symfail}{
    \langle e, xy\rangle \to x
  }
\end{gather*}

\begin{align*}
  \semanticf{(\Sigma, N, R, e_0)} &= \semanticf{e_0} \\
  \semanticf{e} &= \{s \in \Sigma^* \mid \langle e, s\rangle \to s\}
\end{align*}

\subsection{Equivalence}

\subsubsection{Abbreviations}

\begin{align*}
  \mathop{\&} e &= \mathop{!} (\mathop{!} e) &(\text{and predicate}) \\
  e+ &= e e^* &(\text{positive repetition}) \\
  e? &= e / \epsilon &(\text{optional})
\end{align*}

\subsubsection{Associativity}

\begin{gather*}
  \infer{\semanticf{e_1 / (e_2 / e_3)} = \semanticf{(e_1 / e_2) / e_3}}{}
  \\
  \infer{\semanticf{e_1 (e_2 e_3)} = \semanticf{(e_1 e_2) e_3}}{}
\end{gather*}

\subsubsection{Epsilon}

\begin{gather*}
  \infer{\semanticf{\epsilon / e} = \semanticf{\epsilon}}{}
  \\
  \infer{\semanticf{e \epsilon} = \semanticf{e}}{}
  \hspace{2em}
  \infer{\semanticf{\epsilon e} = \semanticf{e}}{}
\end{gather*}

\subsubsection{Repetition}

\begin{align*}
  M &\Coloneqq e M \mid \epsilon
\end{align*}

\begin{gather*}
  \infer{\semanticf{e^*} = \semanticf{M}}{}
\end{gather*}

\subsection{Producing Analysis}

$s \Coloneqq 0 \mid 1$,$o \Coloneqq s \mid \symfail$

\begin{itemize}
  \item $\epsilon \pto 0$
  \item $\sigma \pto 1$
  \item $\sigma \pto \symfail$
  \item $A \assign e \in R$,$e \pto o$ ならば $A \pto o$
  \item $e_1 \pto 0$,$e_2 \pto 0$ ならば $e_1e_2 \pto 0$
  \item $e_1 \pto 1$,$e_2 \pto s$ ならば $e_1e_2 \pto 1$
  \item $e_1 \pto s$,$e_2 \pto 1$ ならば $e_1e_2 \pto 1$
  \item $e_1 \pto \symfail$ ならば $e_1e_2 \pto \symfail$
  \item $e_1 \pto s$,$e_2 \pto \symfail$ ならば $e_1e_2 \pto \symfail$
  \item $e_1 \pto s$ ならば $e_1 \mathrel{/} e_2 \pto s$
  \item $e_1 \pto \symfail$,$e_2 \pto o$ ならば $e_1 \mathrel{/} e_2 \pto o$
  \item $e \pto 1$ ならば $e^* \pto 1$
  \item $e \pto \symfail$ ならば $e^* \pto \symfail$
  \item $e \pto s$ ならば $\mathop{!} e \pto \symfail$
  \item $e \pto \symfail$ ならば $\mathop{!} e \pto 0$
\end{itemize}

\begin{theorem}
  \hwordspace{}
  \begin{itemize}
    \item $\langle e, x\rangle \to \epsilon$ ならば,$e \pto 0$
    \item $\langle e, xy\rangle \to x$,$x \neq \epsilon$ ならば,$e \pto 1$
    \item $\langle e, x\rangle \symfail$ ならば,$e \pto \symfail$
  \end{itemize}
\end{theorem}

\begin{corollary}
  $e \not\pto o$ ならば,$\langle e, xy\rangle \not\to x$ かつ $\langle e, xy\rangle \not{\symfail}$
\end{corollary}
